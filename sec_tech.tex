\lstset{style=fsharpstyle}

\section{Обзор используемых технологий}
\label{sec:technologies}

\subsection{Ruby on Rails}
\label{sub:technologies:ruby_on_rails}
Ruby on Rails - веб-фреймоворк, написанный на языке Ruby. Ruby on Rails известен тем, что позволяет очень быстро начать разрабатывать приложения. В данном веб-фреймворке есть возможность под названием ``scaffold'', которая позволяет одной командой создавать представления, контроллер и модель для нужного объекта.

Философия Ruby on Rails складывается из двух основных пунктов:

\begin{itemize}
  \item ``Convention over configuration'' - что переводится, как преобладание соглашений над конфигурацией. Это означает, что в экосистеме Ruby on Rails разработчикам предлагается пользоваться общепринятыми соглашениями для того, чтобы ускорить разработку. Например, названия таблиц в БД должны называться множественным числом модели, для которой создаётся таблица. Также в настройках самого фреймворка присутствует множество дефолтных значений. Это не означает, что разработчик не может сконфигурировать приложение по своему желанию. Такая возможность, безусловно, присутствует, однако предлагается пользоваться соглашениями. Это не только ускоряет разработку, но и увеличивает уровень взаимопонимая в команде.
  \item ``DRY'' - что расшифровывается, как ``don't repeat yourself''. Это означает избегать дублирования кода. Желательно писать приложение таким образом, чтобы оно легко масштабировалось, методы и компоненты могли быть переиспользованы в различных местах.
\end{itemize}

В Ruby on Rails используется ORM ActiveRecord. ORM представляет собой прослойку между логической моделью и её представлением в БД. ActiveRecord предоставляет богатый функционал для работы с моделями. ActiveRecord - это также шаблон проектирования. Принцип работы ActiveRecord состоит в том, что в приложении создаётся класс, который отображается на таблицу в БД, так что:

\begin{itemize}
  \item любой объект класса, унаследованного от ActiveRecord::Base отображается на строку в таблице БД;
  \item чтобы создать новую запись в таблице, нужно создать новый валидный объект;
  \item в качестве свойств объекта выступают поля в соответствующей строке БД;
  \item строка в таблице БД изменяется или удаляется, если изменяется или удаляется объект ActiveRecord::Base;
\end{itemize}

Также ActiveRecord предоставляет инструмент манипулирования таблицами посредством миграций - методов, в которых на языке Ruby описываются действия с БД. Для создания столбцов таблиц ActiveRecord предлагает следующие типы:

\begin{itemize}
  \item binary
  \item boolean
  \item date
  \item datetime
  \item decimal
  \item float
  \item integer
  \item bigint
  \item primary\_key
  \item references
  \item string
  \item text
  \item time
  \item timestamp
\end{itemize}

Веб-фреймворки в большинстве своём обладают HTML-процессорами, которые позволяют дополнять HTML-разметку различными вставками, и Ruby on Rails не является исключением. Стандартный HTML-процессор в Ruby on Rails называется ERB. Он позволяет помещать в HTML-разметку Ruby-код, который создаётся в контроллере, тем самым давая возможность разработчикам избежать статических HTML-страниц. Однако помимо ERB существуют и другие HTML-процессоры, которые предоставляют возможность не только дополнять HTML-разметку, но и вводить новый вид разметки веб-страниц со вставками Ruby-кода.

В Ruby on Rails присутствует инструмент для соответствия URL-адреса определённому методу в контроллере. Такой инструмент называется router или маршрутизатор. В проекте, созданном на базе Ruby on Rails, существует файл routes.rb, в который помещаются так называемые маршруты. В маршрутизаторе есть возможность не только создавать маршруты не только по одному, но и целыми группами с помощью методы resources, который предлагает создание всех необходимый операций для ресурса: CREATE, READ, UPDATE, DELETE - посредством HTTP запросов.

Выбор фреймворка Ruby on Rails для разработки обусловлен удобством его использования, расширяемостью его с помощью различных библиотек, созданных Ruby on Rails-сообществом, которое на протяжении не менее десяти лет активно развивается и предлагает богатый выбор инструментов для разработки.

\subsection{AngularJS}
\label{sub:technologies:angular_js}
AngularJS - javascript-фреймворк, созданный в компании Google, совершивший прорыв в веб-разработке. AngularJS позволяет создавать динамические SPA. AngularJS, как и Ruby on Rails, обладает MVC-архитектурой. На момент создания AngularJS эта архитектура была революционной по причине того, что javascript-код преимущественно представлял отдельные неорганизованные скрипты, которые очень сложно масштабировать. AngularJS в свою очередь предлагает разделить код на контроллеры, представления и модели(ресурсы), которые в свою очередь могут находиться в отдельных модулях.

AngularJS взаимодействует с HTML, который содержит пользовательские атрибуты, описывающиеся директивами, и соединяет переменные ввода и вывода с помощью объекта JavaScript, значения которых либо задаются вручную, либо извлекаются из некоторых JSON-данных.

AngularJS придерживается философии о том, что для проектирования пользовательских интерфейсов лучше всего подходит декларативное программирование, а для построения бизнес-логики больше подходит императивное программирование.

Фреймворк AngularJS предлагает расширение обыкновенного HTML для обеспечения двусторонней привязки данных, которая позволяет синхронизировать модели (объекты, связанные с бизнес-логикой) и HTML-код. AngularJS упрощает тестирование и позволяет меньше работать с DOM-объектами.

Цели разработки на AngularJS:

\begin{itemize}
  \item отделение логики приложения от манипуляций с DOM-деревом;
  \item поощрение тестирования;
  \item отделение серверной стороны приложения от клиентской, что позволяет разрабатывать эти стороны параллельно;
\end{itemize}

Также AngularJS использует такой паттерн как "внедрение зависимостей", который упрощает использование компонентов приложения тем, что нет необходимости каждый раз создавать объекты классов. Внедряя компонент, система сама заботится о том, чтобы был создан объект нужного класса.

Описывая директивы AngularJS, можно сказать, что они позволяют давать HTML-тегам некоторое поведение. Директивы бывают пользовательские и встроенные. Опишем вкратце некоторые встроенные директивы:

\begin{itemize}
  \item ng-app - делает элемент, к которому применен, корневым для этого приложения;
  \item ng-class - позволяет динамически добавлять классы к HTML-элементам;
  \item ng-show/ng-hide - позволяет показывать/скрывать элемент на основе результата логического выражения;
  \item ng-if - добавляет или удаляет элемент из DOM-дерева на основе результата логического выражения;
  \item ng-bind - заменяет текст, содержащийся в элементе, на результат переданного выражения;
  \item ng-model - позволяет обеспечить двустороннее связывание данных;
  \item ng-repeat - аналог foreach, позволяющий проитерироваться по коллекции и создать DOM-элемент для каждого элемента в коллекции;
\end{itemize}

AngularJS выбран по причине удобства использования, скорости работы в сравнении с такими библиотеками, как jQuery, в которых отсутствует компонентный подход, MVC-архитектура и т.д.

\subsection{Bootstrap}
\label{sub:technologies:bootstrap}
Bootstrap - это CSS-фреймворк, который помогает разрабатывать фронтенд приложения.

До создания этого фреймворка разработчикам приходилось писать CSS-код, самим придумывая некоторую структурированность. Но с приходом Bootstrap разработчики могут использовать встроенные значки, которые называются glyphicons, сетку элементов, размещая HTML-элементы без использования табличной вёрски и свойств типа margin, align и т.д., встроенные классы для придания элементам красивого вида и многие другие возможности.

Одним из главных преимуществ Bootstrap является сетка расположения элементов. Она работает таким образом, что элементы распологаются в таблице, строка за строкой. Строки обозначаются классом "row", и каждая строка разделяется на 12 столбцов, которые могут быть объединены. Для обозначения необходимого колчества столбцов нужно элементу div написать класс "col-y-x", где y - это предполагаемый размер экрана и принимающий значения "sm", "md", "lg", а x - это количество необходимых ячеек от 1 до 12.
