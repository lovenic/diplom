\newcommand{\byr}{Br}

\section{Технико-экономическое обоснование эффективности сервиса, предоставляющего возможность получения новинок музыкальной индустрии}

% Begin Calculations

\FPeval{\totalProgramSize}{8980}
\FPeval{\totalProgramSizeCorrected}{8400}

\FPeval{\normativeManDays}{224}

\FPeval{\additionalComplexity}{0.08}
\FPeval{\complexityFactor}{clip(1 + \additionalComplexity)}

\FPeval{\stdModuleUsageFactor}{0.7}
\FPeval{\originalityFactor}{0.9}

\FPeval{\adjustedManDaysExact}{clip( \normativeManDays * \complexityFactor * \stdModuleUsageFactor * \originalityFactor )}
\FPround{\adjustedManDays}{\adjustedManDaysExact}{0}

\FPeval{\daysInYear}{365}
\FPeval{\redLettersDaysInYear}{5}
\FPeval{\weekendDaysInYear}{118}
\FPeval{\vocationDaysInYear}{24}
\FPeval{\workingDaysInYear}{ clip( \daysInYear - \redLettersDaysInYear - \weekendDaysInYear - \vocationDaysInYear ) }

\FPeval{\developmentTimeMonths}{3}
\FPeval{\developmentTimeYears}{0.25}
\FPeval{\developmentTimeYearsExact}{clip(\developmentTimeMonths / 12)}
\FPround{\developmentTimeYears}{\developmentTimeYearsExact}{2}
\FPeval{\requiredNumberOfProgrammersExact}{ clip( \adjustedManDays / (\developmentTimeYears * \workingDaysInYear) + 0.5 ) }

% тут должно получаться 2 ))
\FPtrunc{\requiredNumberOfProgrammers}{\requiredNumberOfProgrammersExact}{0}

\FPeval{\tariffRateFirst}{250}
\FPeval{\tariffFactorFst}{3.04}
\FPeval{\tariffFactorScnd}{2.84}
\FPeval{\tariffFactorMjr}{3.25}


\FPeval{\employmentFstExact}{clip( \adjustedManDays / \requiredNumberOfProgrammers )}
\FPeval{\employmentFst}{127}
\FPeval{\employmentScnd}{33}
\FPeval{\employmentMjr}{64}

\FPeval{\workingHoursInMonth}{160}
\FPeval{\salaryPerHourMjrExact}{\tariffRateFirst * \tariffFactorMjr / \workingHoursInMonth}
\FPeval{\salaryPerHourFstExact}{\tariffRateFirst * \tariffFactorFst / \workingHoursInMonth}
\FPeval{\salaryPerHourScndExact}{\tariffRateFirst * \tariffFactorScnd / \workingHoursInMonth}
\FPround{\salaryPerHourMjr}{\salaryPerHourMjrExact}{0}
\FPround{\salaryPerHourFst}{\salaryPerHourFstExact}{0}
\FPround{\salaryPerHourScnd}{\salaryPerHourScndExact}{0}

\FPeval{\bonusRate}{1.5}
\FPeval{\workingHoursInDay}{8}
\FPeval{\totalSalaryExact}{\workingHoursInDay * \bonusRate * ( \salaryPerHourFst * \employmentFst + \salaryPerHourScnd * \employmentScnd )}
\FPround{\totalSalary}{\totalSalaryExact}{0}

\FPeval{\additionalSalaryNormative}{20}

\FPeval{\additionalSalaryExact}{clip( \totalSalary * \additionalSalaryNormative / 100 )}
\FPround{\additionalSalary}{\additionalSalaryExact}{0}

\FPeval{\socialNeedsNormative}{0.6}
\FPeval{\socialProtectionNormative}{34}
\FPeval{\socialProtectionFund}{ clip(\socialNeedsNormative + \socialProtectionNormative) }

\FPeval{\socialProtectionCostExact}{clip( (\totalSalary + \additionalSalary) * \socialProtectionFund / 100 )}
\FPround{\socialProtectionCost}{\socialProtectionCostExact}{0}

\FPeval{\taxWorkProtNormative}{4}
\FPeval{\taxWorkProtCostExact}{clip( (\totalSalary + \additionalSalary) * \taxWorkProtNormative / 100 )}
\FPround{\taxWorkProtCost}{\taxWorkProtCostExact}{0}
\FPeval{\taxWorkProtCost}{0} % это считать не нужно, зануляем чтобы не менять формулы

\FPeval{\stuffNormative}{3}
\FPeval{\stuffCostExact}{clip( \totalSalary * \stuffNormative / 100 )}
\FPeval{\stuffCost}{\stuffCostExact}

\FPeval{\timeToDebugCodeNormative}{15}
\FPeval{\reducingTimeToDebugFactor}{0.3}
\FPeval{\adjustedTimeToDebugCodeNormative}{ clip( \timeToDebugCodeNormative * \reducingTimeToDebugFactor ) }

\FPeval{\oneHourMachineTimeCost}{0.5}

\FPeval{\machineTimeCostExact}{ clip( \oneHourMachineTimeCost * \totalProgramSizeCorrected / 100 * \adjustedTimeToDebugCodeNormative ) }
\FPround{\machineTimeCost}{\machineTimeCostExact}{0}

\FPeval{\businessTripNormative}{7}
\FPeval{\businessTripCostExact}{ clip( \totalSalary * \businessTripNormative / 100 ) }
\FPround{\businessTripCost}{\businessTripCostExact}{0}

\FPeval{\otherCostNormative}{10}
\FPeval{\otherCostExact}{clip( \totalSalary * \otherCostNormative / 100 )}
\FPround{\otherCost}{\otherCostExact}{0}

\FPeval{\overheadCostNormative}{8}
\FPeval{\overallCostExact}{clip( \totalSalary * \overheadCostNormative / 100 )}
\FPround{\overheadCost}{\overallCostExact}{0}

\FPeval{\overallCost}{clip( \totalSalary + \additionalSalary + \socialProtectionCost + \taxWorkProtCost + \stuffCost + \machineTimeCost + \businessTripCost + \otherCost + \overheadCost ) }

\FPeval{\supportNormative}{30}
\FPeval{\softwareSupportCostExact}{clip( \overallCost * \supportNormative / 100 )}
\FPround{\softwareSupportCost}{\softwareSupportCostExact}{0}


\FPeval{\baseCost}{ clip( \overallCost + \softwareSupportCost ) }

\FPeval{\profitability}{35}
\FPeval{\incomeExact}{clip( \baseCost / 100 * \profitability )}
\FPround{\income}{\incomeExact}{0}

\FPeval{\estimatedPrice}{clip( \income + \baseCost )}

\FPeval{\localRepubTaxNormative}{3.9}
\FPeval{\localRepubTaxExact}{clip( \estimatedPrice * \localRepubTaxNormative / (100 - \localRepubTaxNormative) )}
\FPround{\localRepubTax}{\localRepubTaxExact}{0}
\FPeval{\localRepubTax}{0}

\FPeval{\ndsNormative}{20}
\FPeval{\ndsExact}{clip( (\estimatedPrice + \localRepubTax) / 100 * \ndsNormative )}
\FPround{\nds}{\ndsExact}{0}


\FPeval{\sellingPrice}{clip( \estimatedPrice + \localRepubTax + \nds )}

\FPeval{\taxForIncome}{18}
\FPeval{\incomeWithTaxes}{clip(\income * (1 - \taxForIncome / 100))}
\FPround\incomeWithTaxes{\incomeWithTaxes}{0}


\FPeval{\fundsExploationSoftware}{10}
\FPeval{\allCostsSoftware}{clip(\sellingPrice + \softwareSupportCost + \fundsExploationSoftware)}
\FPeval{\baseStopTime}{30}
\FPeval{\newStopTime}{8}
\FPeval{\stopHourPrice}{5}
\FPeval{\monthlyProgPayment}{1400}
\FPeval{\paymentUpCoeff}{1.5}
\FPeval{\averageWorkingDaysMontly}{21}
\FPeval{\typicalIssueSolvingInYear}{3800}
\FPeval{\jobsVolume}{3800}
\FPeval{\baseAverageLabourIntensity}{1.3}
\FPeval{\newAverageLabourIntensity}{0.4}

\FPeval{\paymentEconomyOnOneTask}{7.5}
\FPeval{\paymentEconomyByUsingNewSoftware}{28500}
\FPeval{\paymentEconomyCountingUp}{42750}
\FPeval{\paymentEconomyByReducingStopTime}{2483.3}
\FPeval{\commonPaymentEconomyByUsingNewSoftware}{clip(\paymentEconomyCountingUp + \paymentEconomyByReducingStopTime)}
\FPeval{\pureIncome}{clip(\commonPaymentEconomyByUsingNewSoftware - (\commonPaymentEconomyByUsingNewSoftware * \taxForIncome / 100))}
\FPround{\pureIncome}{\pureIncome}{1}

\FPeval{\reductionCoeffFirst}{1.000}
\FPeval{\reductionCoeffSecond}{0.869}
\FPeval{\reductionCoeffThird}{0.756}
\FPeval{\reductionCoeffFourth}{0.659}

\FPeval{\overcomeValueSecond}{clip(\reductionCoeffSecond * \pureIncome)}
\FPround{\overcomeValueSecond}{\overcomeValueSecond}{1}
\FPeval{\growingResultSecond}{clip(-\allCostsSoftware + \overcomeValueSecond)}
\FPround{\growingResultSecond}{\growingResultSecond}{1}
\FPeval{\overcomeValueThird}{clip(\reductionCoeffThird * \pureIncome)}
\FPround{\overcomeValueThird}{\overcomeValueThird}{1}
\FPeval{\growingResultThird}{clip(\growingResultSecond + \overcomeValueThird)}
\FPround{\growingResultThird}{\growingResultThird}{1}
\FPeval{\overcomeValueFourth}{clip(\reductionCoeffThird * \pureIncome)}
\FPround{\overcomeValueFourth}{\overcomeValueFourth}{1}
\FPeval{\growingResultFourth}{clip(\growingResultThird + \overcomeValueThird)}
\FPround{\growingResultFourth}{\growingResultFourth}{1}
% End Calculations

\subsection{Введение и исходные данные}

Целью данной дипломной работы является создание программного продукта, позволяющего пользователю получать новинки музыкальной индустрии в виде уведомлений о недавно вышедших публикациях выбранных им музыкальных исполнителей. Данный программный продукт позволит получать информацию из нескольких источников, тем самым помогая пользователю удобнее ориентироваться среди различных сервисов, получать только необходимую и интересную информацию. Это очень эффективно, поскольку сокращается время на поиски новых публикаций самостоятельно. В экономическом смысле - это сократит расходы на подписки на несколько сервисов, которые занимаются размещением на своей платформе новых музыкальных публикаций.

Для разработчика экономическая эффективность будет определяться как прибыль от реализации, а для пользователя - как экономия затрат. Для определения экономической эффективности рассчитаем смету затрат и цену программного продукта.

\subsection{Расчет сметы затрат и цены программного продукта}

Для представления программного продукта на рынке он должен быть законченным, иметь презентабельный вид. Для реализации программного продукта необходимо пройти через два этапа: этап, связанный с разработкой ПО(выяснение требований, программирование, тестирование, отладка), и этап, связанный с непосредственной реализацией продукта на рынке (реализация, поддержка).

Программный комплекс относится ко 2-й группе сложности. Категория новизны – “В”. Расчеты выполнены на основе методического пособия [1].

Исходные данные для проекта указаны в таблице~\ref{table:econ:initial_data}.

\begin{table}
\caption{Исходные данные}
\label{table:econ:initial_data}
  \centering
  \begin{tabular}{| >{\centering}m{0.3\textwidth}
                  | >{\centering}m{0.2\textwidth}
                  | >{\centering}m{0.2\textwidth}
                  | >{\centering\arraybackslash}m{0.2\textwidth}|}
    \hline
    Наименование показателей & Буквенные обозначения & Единицы измерения & Количество \\
    \hline
    Коэффициент новизны & $ \text{К}_\text{н} $ & единиц & \num{\originalityFactor} \\
    \hline
    Группа сложности & & единиц & 2 \\
    \hline
    Дополнительный коэффициент сложности & $ \text{К}_\text{сл} $ & единиц & \num{\complexityFactor} \\
    \hline
    Поправочный коэффициент, учитывающий использование типовых программ & $ \text{К}_\text{т} $ & единиц & \num{\stdModuleUsageFactor} \\
    \hline
    Установленная плановая продолжительность разработки & $ \text{T}_\text{р} $ & лет & \num{\developmentTimeYears} \\
    \hline
    Продолжительность рабочего дня & $ \text{T}_\text{ч} $ & часов & \num{\workingHoursInDay} \\
    \hline
    Тарифная ставка 1-го разряда & $ \text{T}_\text{м1} $ & руб. & \num{\tariffRateFirst} \\
    \hline
    Коэффициент премирования & $ \text{К}_\text{п} $ & единиц & \num{\bonusRate} \\
    \hline
    Норматив дополнительной заработной платы исполнителей & $ \text{Н}_\text{д} $ & \% & \num{\additionalSalaryNormative} \\
    \hline
    Отчисления в фонд социальной защиты населения & $ \text{З}_\text{сз} $ & \% & \num{\socialProtectionNormative} \\
    \hline
    Отчисления в Белгосстрах & $ \text{Н}_\text{нс} $ & \% & \num{\socialNeedsNormative} \\
    \hline
    Расходы на научные командировки & $ \text{Р}_\text{нк} $ & \% & \num{\businessTripNormative} \\
    \hline
    Прочие прямые расходы & $ \text{П}_\text{з} $ & \% & \num{\otherCostNormative} \\
    \hline
    Прочие накладные расходы & $ \text{П}_\text{рн} $ & \% & \num{\overheadCostNormative} \\
    \hline
    Налог на прибыль & $ \text{П}_\text{рн} $ & \% & \num{\taxForIncome} \\
    \hline
  \end{tabular}
\end{table}

Объем программного средства определяется путем подбора аналогов на основании классификации типов программного средства, каталога функций, которые постоянно обновляются и утверждаются в установленном порядке. На основании инфорациии и функциях разрабатываемого программного средства по каталогу функций определяется объем функций. Объем программного средства определяется на основе нормативных данных, приведенных в таблице~\ref{table:econ:volume_of_po}.

\begin{table}
\caption{Перечень и объем функций программного модуля}
\label{table:econ:volume_of_po}
\centering
  \begin{tabular}{| >{\centering}m{0.12\textwidth}
                  | >{\centering}m{0.40\textwidth}
                  | >{\centering}m{0.18\textwidth}
                  | >{\centering\arraybackslash}m{0.18\textwidth}|}
  \hline
         \multirow{2}{0.12\textwidth}[-0.5em]{\centering \No{} функции}
       & \multirow{2}{0.40\textwidth}[-0.55em]{\centering Наименование (содержание)}
       & \multicolumn{2}{c|}{\centering Объем функции, LoC} \tabularnewline
  \cline{3-4} &
       & { по каталогу ($ V_{i} $) }
       & { уточненный ($ V_{i}^{\text{у}} $) } \tabularnewline
  \hline
  101 & Организация ввода информации & \num{150} & \num{200} \tabularnewline
  \hline
  201 & Генерация структуры базы данных & \num{4300} & \num{3560} \tabularnewline
  \hline
  204 & Обработка наборов и записей базы данных & \num{2670} & \num{2310} \tabularnewline
  \hline
  506 & Обработка ошибочных и сбойных ситуаций & \num{410} & \num{560} \tabularnewline
  \hline
  507 & Обеспечение интерфейсов между компонентами & \num{970} & \num{1020} \tabularnewline
  \hline
  707 & Графический вывод результатов & \num{480} & \num{750} \tabularnewline
  \hline
  & Всего & \num{\totalProgramSize} & \num{\totalProgramSizeCorrected} \tabularnewline
  \hline
  \end{tabular}
\end{table}

Общий объем программного продукта определяется исходя из количества и объема функций, реализованных в программе, рассчитывается по формуле~\ref{eq:econ:total_program_size}:
\begin{equation}
  \label{eq:econ:total_program_size}
  V_{o} = \sum_{i = 1}^{n} V_{i} \text{\,,}
\end{equation}
\begin{explanation}
где & $ V_{i} $ & объем отдельной функции ПО, LoC; \\
    & $ n $ & общее число функций.
\end{explanation}

Уточненный объем ПО ($ V_{o} $) рассчитывается по формуле~\ref{eq:econ:total_program_size_corrected}:
\begin{equation}
  \label{eq:econ:total_program_size_corrected}
  V_{\text{у}} = \sum_{i = 1}^{n} V_{i}^{\text{у}} \text{\,,}
\end{equation}
\begin{explanation}
где & $ V_{i}^{\text{y}} $ & уточненный объем отдельной функции ПО, LoC; \\
    & $ n $ & общее число функций.
\end{explanation}

Уточненный объем ПО - 8400 LOC. По отнесено ко второй категории сложности: предполагается обеспечить переносимость ПО. Наличие характеристики, определяющей сложность ПО, позволяет применить к объему ПО коэффициент $ \text{К}_\text{с} $. Посредством коэффициента сложности учитываются дополнительные затраты труда, связанные со сложностью разрабатываемого программного продукта. Коэффициент сложности рассчитывается по формуле~\ref{eq:econ:complexity_coeff}:

\begin{equation}
\label{eq:econ:complexity_coeff}
  \text{К}_{\text{с}} = 1 + \sum_{i = 1}^n \text{К}_{i} \text{\,,}
\end{equation}
\begin{explanation}
где & $ \text{К}_{i} $ & коэффициент, соответствующий степени повышения сложности ПО за счет конкретной характеристики; \\
    & $ n $ & количество учитываемых характеристик.
\end{explanation}

Наличие двух характеристик сложности позволяет~\cite[c.~66, приложение~4, таблица~П.4.2]{palicyn_2006} вычислить коэффициент сложности:
\begin{equation}
\label{eq:econ:complexity_coeff_calc}
  \text{К}_{\text{с}} = \num{1} + \num{\additionalComplexity} = \num{\complexityFactor} \text{\,.}
\end{equation}

Коэффициент использования стандартных модулей $ \text{К}_\text{т} $ = \num{\stdModuleUsageFactor}, а коэффициент новизны ПО $ \text{К}_\text{н} $ = \num{\originalityFactor}

Нормативная трудоемкость разработки ПО ($ \text{Т}_\text{н} $) определяется согласно прил.3 (гр.1, стр. 20 - 8400 LOC; гр.3 стр. 20 - \num{\normativeManDays} чел./дн.) и составляет \num{\normativeManDays} чел./дн.
Нормативная трудоемкость служит основой для оценки общей трудоемкости~$ \text{Т}_\text{о} $.
Используем формулу (\ref{eq:econ:effort_common}) для оценки общей трудоемкости для небольших проектов:
\begin{equation}
  \label{eq:econ:effort_common}
  \text{Т}_\text{о} = \text{Т}_\text{н} \cdot
                      \text{К}_\text{с} \cdot
                      \text{К}_\text{т} \cdot
                      \text{К}_\text{н} \text{\,,}
\end{equation}
\begin{explanation}
где & $ \text{К}_\text{с} $ & коэффициент, учитывающий сложность ПО; \\
    & $ \text{К}_\text{т} $ & поправочный коэффициент, учитывающий степень использования при разработке стандартных модулей; \\
    & $ \text{К}_\text{н} $ & коэффициент, учитывающий степень новизны ПО.
\end{explanation}

Следовательно общая трудоемкость составит:
\begin{equation}
  \label{eq:econ:effort_common_calc}
  \text{Т}_\text{о} = \num{\normativeManDays} \times \num{\complexityFactor} \times \num{\stdModuleUsageFactor} \times \num{\originalityFactor} \approx \SI{\adjustedManDays}{\text{чел.}/\text{дн.}}
\end{equation}

На основе общей трудоемкости и требуемых сроков реализации проекта вычисляется плановое количество исполнителей.
Численность исполнителей проекта рассчитывается по формуле~\ref{eq:econ:num_of_programmers}:
\begin{equation}
  \label{eq:econ:num_of_programmers}
  \text{Ч}_\text{р} = \frac{\text{Т}_\text{о}}{\text{Т}_\text{р} \cdot \text{Ф}_\text{эф}} \text{\,,}
\end{equation}
\begin{explanation}
где & $ \text{Т}_\text{о} $ & общая трудоемкость разработки проекта, $ \text{чел.}/\text{дн.} $; \\
    & $ \text{Ф}_\text{эф} $ & эффективный фонд времени работы одного работника в течение года, дн.; \\
    & $ \text{Т}_\text{р} $ & срок разработки проекта, лет.
\end{explanation}

Эффективный фонд времени работы одного разработчика вычисляется по формуле~\ref{eq:econ:effective_time_per_programmer}:
\begin{equation}
  \label{eq:econ:effective_time_per_programmer}
  \text{Ф}_\text{эф} =
    \text{Д}_\text{г} -
    \text{Д}_\text{п} -
    \text{Д}_\text{в} -
    \text{Д}_\text{о} \text{\,,}
\end{equation}
\begin{explanation}
где & $ \text{Д}_\text{г} $ & количество дней в году, дн.; \\
    & $ \text{Д}_\text{п} $ & количество праздничных дней в году, не совпадающих с выходными днями, дн.; \\
    & $ \text{Д}_\text{в} $ & количество выходных дней в году, дн.; \\
    & $ \text{Д}_\text{п} $ & количество дней отпуска, дн.
\end{explanation}

Согласно данным, приведенным в производственном календаре для пятидневной рабочей недели в 2017 году для Беларуси~\cite{belcalendar_2013}, фонд рабочего времени составит:
\begin{equation}
  \text{Ф}_\text{эф} = \num{\daysInYear} - \num{\redLettersDaysInYear} - \num{\weekendDaysInYear} - \num{\vocationDaysInYear} = \SI{\workingDaysInYear}{\text{дн.}}
\end{equation}

Учитывая срок разработки проекта $ \text{Т}_\text{р} = \SI{\developmentTimeMonths}{\text{мес.}} = \SI{\developmentTimeYears}{\text{года}} $, общую трудоемкость и фонд эффективного времени одного работника, вычисленные ранее, можем рассчитать численность исполнителей проекта:
\begin{equation}
  \label{eq:econ:num_of_programmers_calc}
  \text{Ч}_\text{р} =
    \frac{\num{\adjustedManDays}}
         {\num{\developmentTimeYears} \times \num{\workingDaysInYear}}
    \approx \SI{\requiredNumberOfProgrammers}{\text{рабочих}}.
\end{equation}

Вычисленные оценки показывают, что для выполнения запланированного проекта в указанные сроки необходимо три рабочих.
Информация о работниках перечислена в таблице~\ref{table:econ:programmers}.
\begin{table}[ht]
  \caption{Работники, занятые в проекте}
  \label{table:econ:programmers}
  \begin{tabular}{| >{\centering}m{0.4\textwidth}
                  | >{\centering}m{0.15\textwidth}
                  | >{\centering}m{0.18\textwidth}
                  | >{\centering\arraybackslash}m{0.15\textwidth}|}
   \hline
   Исполнители & Разряд & Тарифный коэффициент & \mbox{Чел./дн.} занятости \\
   \hline
   Программист \Rmnum{2}-категории & $ \num{12} $ & $ \num{\tariffFactorScnd} $ & $ \num{\employmentScnd} $ \\
   \hline
   Программист \Rmnum{1}-категории & $ \num{13} $ & $ \num{\tariffFactorFst} $ & $ \num{\employmentFst} $ \\
   \hline
   Ведущий программист & $ \num{14} $ & $ \num{\tariffFactorMjr} $ & $ \num{\employmentMjr} $ \\
   \hline
  \end{tabular}
\end{table}

Месячная тарифная ставка одного работника вычисляется по формуле~\ref{eq:econ:month_salary}:
\begin{equation}
  \label{eq:econ:month_salary}
  \text{Т}_\text{ч} =
    \frac {\text{Т}_{\text{м}_{1}} \cdot \text{Т}_{\text{к}} }
          {\text{Ф}_{\text{р}} }  \text{\,,}
\end{equation}
\begin{explanation}
где & $ \text{Т}_{\text{м}_{1}} $ & месячная тарифная ставка 1-го разряда, \byr; \\
    & $ \text{Т}_{\text{к}} $ & тарифный коэффициент, соответствующий установленному тарифному разряду; \\
    & $ \text{Ф}_{\text{р}} $ & среднемесячная норма рабочего времени, час.
\end{explanation}

Подставив данные из таблицы~\ref{table:econ:programmers} в формулу~(\ref{eq:econ:month_salary}), приняв значение тарифной ставки 1-го разряда $ \text{Т}_{\text{м}_{1}} = \SI{\tariffRateFirst}{\text{\byr}} $ и среднемесячную норму рабочего времени $ \text{Ф}_{\text{р}} = \SI{\workingHoursInMonth}{\text{часов}} $ получаем
\begin{equation}
  \label{eq:econ:month_salary_calc1}
  \text{Т}_{\text{ч}}^{\text{прогр. \Rmnum{2}-разр.}} = \frac{ \num{\tariffRateFirst} \times \num{\tariffFactorScnd} } { \num{\workingHoursInMonth} } = \SI{\salaryPerHourScnd}{\text{\byr}/\text{час;}}
\end{equation}
\begin{equation}
  \label{eq:econ:month_salary_calc2}
  \text{Т}_{\text{ч}}^{\text{прогр. \Rmnum{1}-разр.}} = \frac{ \num{\tariffRateFirst} \times \num{\tariffFactorFst} } { \num{\workingHoursInMonth} } = \SI{\salaryPerHourFst}{\text{\byr}/\text{час;}}
\end{equation}
\begin{equation}
  \label{eq:econ:month_salary_calc3}
  \text{Т}_{\text{ч}}^{\text{вед. прогр.}} = \frac{ \num{\tariffRateFirst} \times \num{\tariffFactorMjr} } { \num{\workingHoursInMonth} } = \SI{\salaryPerHourMjr}{\text{\byr}/\text{час.}}
\end{equation}

Основная заработная плата исполнителей на конкретное ПО рассчитывается по формуле
\begin{equation}
  \label{eq:econ:total_salary}
  \text{З}_{\text{о}} = \sum^{n}_{i = 1}
                        \text{Т}_{\text{ч}}^{i} \cdot
                        \text{Т}_{\text{ч}} \cdot
                        \text{Ф}_{\text{п}} \cdot
                        \text{К}
                          \text{\,,}
\end{equation}
\begin{explanation}
где & $ \text{Т}_{\text{ч}}^{i} $ & часовая тарифная ставка \mbox{$ i $-го} исполнителя, \byr$/$час; \\
    & $ \text{Т}_{\text{ч}} $ & количество часов работы в день, час; \\
    & $ \text{Ф}_{\text{п}} $ & плановый фонд рабочего времени \mbox{$ i $-го} исполнителя, дн.; \\
    & $ \text{К} $ & коэффициент премирования.
\end{explanation}

Подставив ранее вычисленные значения и данные из таблицы~\ref{table:econ:programmers} в формулу~(\ref{eq:econ:total_salary}) и приняв коэффициент премирования $ \text{К} = \num{\bonusRate} $ получим
\begin{equation}
  \label{eq:econ:total_salary_calc}
  \text{З}_{\text{о}} = (\salaryPerHourMjr \times \num{\employmentMjr} +\salaryPerHourFst \times \num{\employmentFst} + \salaryPerHourScnd \times \num{\employmentScnd}) \times \num{\workingHoursInDay} \times \num{\bonusRate} = \SI{\totalSalary}{\text{\byr}} \text{\,.}
\end{equation}

Дополнительная заработная плата включает выплаты предусмотренные законодательством от труде и определяется по нормативу в процентах от основной заработной платы
\begin{equation}
  \label{eq:econ:additional_salary}
  \text{З}_{\text{д}} =
    \frac {\text{З}_{\text{о}} \cdot \text{Н}_{\text{д}}}
          {100\%} \text{\,,}
\end{equation}
\begin{explanation}
  где & $ \text{Н}_{\text{д}} $ & норматив дополнительной заработной платы, $ \% $.
\end{explanation}

Приняв норматив дополнительной заработной платы $ \text{Н}_{\text{д}} = \num{\additionalSalaryNormative\%} $ и подставив известные данные в формулу~(\ref{eq:econ:additional_salary}) получим
\begin{equation}
  \label{eq:econ:additional_salary_calc}
  \text{З}_{\text{д}} =
    \frac{\num{\totalSalary} \times 20\%}
         {100\%} \approx \SI{\additionalSalary}{\text{\byr}} \text{\,.}
\end{equation}

Согласно действующему законодательству отчисления в фонд социальной защиты населения составляют \num{\socialProtectionNormative\%} , в фонд обязательного страхования "--- \num{\socialNeedsNormative\%}, от фонда основной и дополнительной заработной платы исполнителей.
Общие отчисления на социальную защиту рассчитываются по формуле
\begin{equation}
  \label{eq:econ:soc_prot}
  \text{З}_{\text{сз}} =
    \frac{(\text{З}_{\text{о}} + \text{З}_{\text{д}}) \cdot \text{Н}_{\text{сз}}}
         {\num{100\%}} \text{\,.}
\end{equation}

Подставив вычисленные ранее значения в формулу~(\ref{eq:econ:soc_prot}) получаем
\begin{equation}
  \label{eq:econ:soc_prot_calc}
  \text{З}_{\text{сз}} =
    \frac{ (\num{\totalSalary} + \num{\additionalSalary}) \times \num{\socialProtectionFund\%} }
         { \num{100\%} }
    \approx \SI{\socialProtectionCost}{\text{\byr}} \text{\,.}
\end{equation}

\begin{comment}
  Расчет налогов от фонда оплаты труда производится формуле
  \begin{equation}
    \label{eq:econ:tax_work_prot}
    \text{Н}_{\text{е}} =
      \frac{(\text{З}_{\text{о}} + \text{З}_{\text{д}}) \cdot \text{Н}_{\text{не}}}
           {\num{100\%}} \text{\,,}
  \end{equation}
  \begin{explanation}
    где & $ \text{Н}_{\text{не}} $ & норматив налога, уплачиваемый единым платежом, $ \% $.
  \end{explanation}

  Подставив ранее вычисленные значения в формулу~(\ref{eq:econ:tax_work_prot}) и приняв норматив налога $ \text{Н}_{\text{не}} = \num{\taxWorkProtNormative\%} $ получаем
  \begin{equation}
    \label{eq:econ:tax_work_prot_calc}
    \text{Н}_{\text{е}} =
        \frac{ (\num{\totalSalary} + \num{\additionalSalary}) \times \num{\taxWorkProtNormative\%} }
           { \num{100\%} }
      \approx \SI{\taxWorkProtCost}{\text{\byr}}\text{\,.}
  \end{equation}
\end{comment}

По статье <<материалы>> проходят расходы на носители информации, бумагу, краску для принтеров и другие материалы, используемые при разработке ПО.
Норма расходов $ \text{Н}_{\text{мз}} $ определяется либо в расчете на \num{100} строк исходного кода, либо в процентах к основной зарплате исполнителей \mbox{\num{3\%}\,---\,\num{5\%}}.
Затраты на материалы вычисляются по формуле
\begin{equation}
  \label{eq:econ:stuff}
  \text{М} =
    \frac{ \text{З}_{\text{о}} \cdot \text{Н}_{\text{мз}} }
         { \num{100\%} } =
    \frac{ \num{\totalSalary} \times \num{\stuffNormative\%} }
         { \num{100\%} } \approx
    \SI{\stuffCost}{\text{\byr}} \text{\,.}
\end{equation}

Расходы по статье <<машинное время>> включают оплату машинного времени, необходимого для разработки и отладки ПО, которое определяется по нормативам в машино-часах на \num{100} строк исходного кода в зависимости от характера решаемых задач и типа ПК, и вычисляются по формуле
\begin{equation}
  \label{eq:econ:machine_time}
  \text{Р}_{\text{м}} =
    \text{Ц}_{\text{м}} \cdot
    \frac {\text{V}_{\text{о}}}
          {\num{100}} \cdot
    \text{Н}_{\text{мв}} \text{\,,}
\end{equation}
\begin{explanation}
  где & $ \text{Ц}_{\text{м}} $ & цена одного часа машинного времени, \byr; \\
      & $ \text{Н}_{\text{мв}} $ & норматив расхода машинного времени на отладку 100 строк исходного кода, часов.
\end{explanation}

Согласно нормативу~\cite[с.\,69, приложениe~6]{palicyn_2006} норматив расхода машинного времени на отладку \num{100} строк исходного кода составляет $ \text{Н}_{\text{мв}} = \num{\adjustedTimeToDebugCodeNormative} $.
Цена одного часа машинного времени составляет $ \text{Ц}_{\text{м}} = \SI{\oneHourMachineTimeCost}{\text{\byr}} $.
Подставляя известные данные в формулу~(\ref{eq:econ:machine_time}) получаем
\begin{equation}
  \label{eq:econ:machine_time_calc}
  \text{Р}_{\text{м}} =
    \num{\oneHourMachineTimeCost} \times
    \frac {\num{\totalProgramSizeCorrected}}
          {\num{100}} \times
    \num{\adjustedTimeToDebugCodeNormative} =
    \SI{\machineTimeCost}{\text{\byr}} \text{\,.}
\end{equation}

Расходы по статье <<научные командировки>> вычисляются как процент от основной заработной платы, либо определяются по нормативу.
Вычисления производятся по формуле
\begin{equation}
  \label{eq:econ:business_trip}
  \text{Р}_{\text{к}} =
    \frac{ \text{З}_{\text{о}} \cdot \text{Н}_{\text{к}} }
         { \num{100\%} } \text{\,,}
\end{equation}
\begin{explanation}
  где & $ \text{Н}_{\text{к}} $ & норматив командировочных расходов по отношению к основной заработной плате, $ \% $.
\end{explanation}

Подставляя ранее вычисленные значения в формулу~(\ref{eq:econ:business_trip}) и приняв значение $ \text{Н}_{\text{к}} = \num{\businessTripNormative\%} $ получаем
\begin{equation}
  \label{eq:econ:business_trip_calc}
    \text{Р}_{\text{к}} =
    \frac{ \num{\totalSalary} \times \num{\businessTripNormative\%} }
         { \num{100\%} } = \SI{\businessTripCost}{\text{\byr}} \text{\,.}
\end{equation}

Статья расходов <<прочие затраты>> включает в себя расходы на приобретение и подготовку специальной научно-технической информации и специальной литературы.
Затраты определяются по нормативу принятому в организации в процентах от основной заработной платы и вычисляются по формуле
\begin{equation}
  \label{eq:econ:other_cost}
  \text{П}_{\text{з}} =
    \frac{ \text{З}_{\text{о}} \cdot \text{Н}_{\text{пз}} }
         { \num{100\%} } \text{\,,}
\end{equation}
\begin{explanation}
  где & $ \text{Н}_{\text{пз}} $ & норматив прочих затрат в целом по организации, $ \% $.
\end{explanation}

Приняв значение норматива прочих затрат $ \text{Н}_{\text{пз}} = \num{\otherCostNormative\%} $ и подставив вычисленные ранее значения в формулу~(\ref{eq:econ:other_cost}) получаем
\begin{equation}
  \label{eq:econ:other_cost_calc}
  \text{П}_{\text{з}} =
    \frac{ \num{\totalSalary} \times \num{\otherCostNormative\%} }
         { \num{100\%} } =
    \SI{\otherCost}{\text{\byr}} \text{\,.}
\end{equation}

Статья <<накладные расходы>> учитывает расходы, необходимые для содержания аппарата управления, вспомогательных хозяйств и опытных производств, а также расходы на общехозяйственные нужны. Данная статья затрат рассчитывается по нормативу от основной заработной платы и вычисляется по формуле.

\begin{equation}
  \label{eq:econ:overhead_cost}
  \text{Р}_{\text{н}} =
    \frac{ \text{З}_{\text{о}} \cdot \text{Н}_{\text{рн}} }
         { \num{100\%} } \text{\,,}
\end{equation}
\begin{explanation}
  где & $ \text{Н}_{\text{рн}} $ & норматив накладных расходов в организации,~$ \% $.
\end{explanation}

Приняв норму накладных расходов $ \text{Н}_{\text{рн}} = \num{\overheadCostNormative\%} $ и подставив известные данные в формулу~(\ref{eq:econ:overhead_cost}) получаем
\begin{equation}
  \label{eq:econ:overhead_cost_calc}
  \text{Р}_{\text{н}} =
    \frac{ \num{\totalSalary} \times \num{\overheadCostNormative\%} }
         { \num{100\%} } =
    \SI{\overheadCost}{\text{\byr}} \text{\,.}
\end{equation}

Общая сумма расходов по смете на ПО рассчитывается по формуле
\begin{equation}
  \label{eq:econ:overall_cost}
  \text{С}_{\text{р}} =
    \text{З}_{\text{о}} +
    \text{З}_{\text{д}} +
    \text{З}_{\text{сз}} +
    %\text{Н}_{\text{е}} +
    \text{М} +
    % \text{Р}_{\text{с}} + % спецоборудование не нужно
    \text{Р}_{\text{м}} +
    \text{Р}_{\text{нк}} +
    \text{П}_{\text{з}} +
    \text{Р}_{\text{н}}\text{\,.}
\end{equation}

Подставляя ранее вычисленные значения в формулу~(\ref{eq:econ:overall_cost}) получаем

\begin{equation}
  \label{eq:econ:overall_cost_calc}
  \text{С}_{\text{р}} = \SI{\overallCost}{\text{\byr}} \text{\,.}
\end{equation}

Расходы на сопровождение и адаптацию, которые несет производитель ПО, вычисляются по нормативу от суммы расходов по смете и рассчитываются по формуле
\begin{equation}
  \label{eq:econ:software_support}
  \text{Р}_{\text{са}} =
    \frac { \text{С}_{\text{р}} \cdot \text{Н}_{\text{рса}} }
          { \num{100\%} } \text{\,,}
\end{equation}
\begin{explanation}
  где & $ \text{Н}_{\text{рса}} $ & норматив расходов на сопровождение и адаптацию ПО,~$ \% $.
\end{explanation}

Приняв значение норматива расходов на сопровождение и адаптацию $ \text{Н}_{\text{рса}} = \num{\supportNormative\%} $ и подставив ранее вычисленные значения в формулу~(\ref{eq:econ:software_support}) получаем
\begin{equation}
  \label{eq:econ:software_support_calc}
  \text{Р}_{\text{са}} =
    \frac { \num{\overallCost} \times \num{\supportNormative\%} }
          { \num{100\%} } \approx \SI{\softwareSupportCost}{\text{\byr}} \text{\,.}
\end{equation}

Полная себестоимость создания ПО включает сумму затрат на разработку, сопровождение и адаптацию и вычисляется по формуле
\begin{equation}
  \label{eq:econ:base_cost}
  \text{С}_{\text{п}} = \text{С}_{\text{р}} + \text{Р}_{\text{са}} \text{\,.}
\end{equation}

Подставляя известные значения в формулу~(\ref{eq:econ:base_cost}) получаем
\begin{equation}
  \label{eq:econ:base_cost_calc}
  \text{С}_{\text{п}} = \num{\overallCost} + \num{\softwareSupportCost} = \SI{\baseCost}{\text{\byr}} \text{\,.}
\end{equation}

Расчет прогнозируемой прибыли:

\begin{equation}
  \label{eq:econ:margin_calculation}
  \text{П}_{\text{с}} =
    \frac { \text{С}_{\text{п}} \cdot \text{У}_{\text{рп}} }
          { \num{100\%} } \text{\,,}
\end{equation}
\begin{explanation}
  где & $ \text{П}_{\text{с}} $ & прибыль от реализации ПО заказчику, \byr; \\
      & $ \text{У}_{\text{рп}} $ & уровень рентабельности ПО,~$ \% $.
\end{explanation}

Подставив известные данные в формулу~(\ref{eq:econ:income_calc}) получаем прогнозируемую прибыль от реализации ПО
\begin{equation}
  \label{eq:econ:income_calc}
  \text{П}_{\text{с}} =
    \frac { \num{\baseCost} \times \num{\profitability\%} }
          { \num{100\%} }
    \approx \SI{\income}{\text{\byr}} \text{\,.}
\end{equation}

Прогнозируемая цена ПО без учета налогов включаемых в цену вычисляется по формуле
\begin{equation}
  \label{eq:econ:estimated_price}
  \text{Ц}_{\text{п}} = \text{С}_{\text{п}} + \text{П}_{\text{с}}  \text{\,.}
\end{equation}

Подставив данные в формулу~(\ref{eq:econ:estimated_price}) получаем цену ПО без налогов
\begin{equation}
  \label{eq:econ:estimated_price_calc}
  \text{Ц}_{\text{п}} = \num{\baseCost}  + \num{\income} = \SI{\estimatedPrice}{\text{\byr}} \text{\,.}
\end{equation}

\begin{comment}
  Отчисления и налоги в местный и республиканский бюджеты вычисляются по формуле
  \begin{equation}
    \label{eq:econ:local_repub_tax}
    \text{О}_{\text{мр}} =
      \frac { \text{Ц}_{\text{п}} \cdot \text{Н}_{\text{мр}} }
            { \num{100\%} - \text{Н}_{\text{мр}} } \text{\,,}
  \end{equation}
  \begin{explanation}
    где & $ \text{Н}_{\text{мр}} $ & норматив отчислений в местный и республиканский бюджеты, \byr.
  \end{explanation}

  Приняв норматив отчислений в местный и республиканский бюджеты $ \text{Н}_{\text{мр}} = \num{\localRepubTaxNormative\%} $ и подставив известные данные в формулу~(\ref{eq:econ:local_repub_tax}) получим величину единого платежа
  \begin{equation}
    \label{eq:econ:local_repub_tax_calc}
    \text{О}_{\text{мр}} =
      \frac { \num{\estimatedPrice} \cdot \num{\localRepubTaxNormative\%} }
            { \num{100\%} - \num{\localRepubTaxNormative\%} }
      \approx \SI{\localRepubTax}{\text{\byr}} \text{\,.}
  \end{equation}
\end{comment}

Налог на добавленную стоимость рассчитывается по формуле
\begin{equation}
  \label{eq:econ:nds}
  \text{НДС}_{\text{}} =
    \frac{ \text{Ц}_{\text{п}} \cdot \text{Н}_{\text{дс}} }
         { \num{100\%} } \text{\,,}
\end{equation}
\begin{explanation}
  где & $ \text{Н}_{\text{дс}} $ & норматив НДС,~$\%$.
\end{explanation}

Норматив НДС составляет $ \text{Н}_{\text{дс}} = \num{\ndsNormative\%} $, подставляя известные значения в формулу~(\ref{eq:econ:nds}) получаем
\begin{equation}
  \text{НДС} =
    \frac { \num{\estimatedPrice} \times \num{\ndsNormative\%} }
          { \num{100\% }}
    \approx \SI{\nds}{\text{\byr}} \text{\,.}
\end{equation}

Расчет прогнозируемой отпускной цены осуществляется по формуле
\begin{equation}
  \label{eq:econ:selling_price}
  \text{Ц}_{\text{о}} = \text{Ц}_{\text{п}} + \text{НДС} \text{\,.}
\end{equation}

Подставляя известные данные в формулу~(\ref{eq:econ:selling_price}) получаем прогнозируемую отпускную цену
\begin{equation}
  \label{eq:econ:selling_price_calc}
  \text{Ц}_{\text{о}} = \num{\estimatedPrice} + \num{\nds} \approx \SI{\sellingPrice}{\text{\byr}} \text{\,.}
\end{equation}


Чистую прибыль от реализации проекта можно рассчитать по формуле
\begin{equation}
  \label{eq:econ:income_with_taxes}
  \text{П}_\text{ч} =
    \text{П}_\text{c} \cdot
    \left(1 - \frac{ \text{Н}_\text{п} }{ \num{100\%} } \right) \text{\,,}
\end{equation}
\begin{explanation}
  где & $ \text{Н}_{\text{п}} $ & величина налога на прибыль,~$\%$.
\end{explanation}

Приняв значение налога на прибыль $ \text{Н}_{\text{н}} = \num{\taxForIncome\%} $ и подставив известные данные в формулу~(\ref{eq:econ:income_with_taxes}) получаем чистую прибыль
\begin{equation}
  \label{eq:econ:income_with_taxes_calc}
  \text{П}_\text{ч} =
    \num{\income} \times \left( 1 - \frac{\num{\taxForIncome\%}}{100\%} \right) = \SI{\incomeWithTaxes}{\text{\byr}} \text{\,.}
\end{equation}

Результаты расчета себестоимости и отпускной цены представлены в таблице~\ref{table:econ:selfprice}

\begin{table}
\caption{Себестоимость и отпускная цена ПО}
\label{table:econ:selfprice}
  \centering
  \begin{tabular}{| >{\centering}m{0.55\textwidth}
                  | >{\centering}m{0.15\textwidth}
                  | >{\centering\arraybackslash}m{0.15\textwidth}|}
    \hline
    {\begin{center}
      Наименование
    \end{center} } & Условное обозначение & Значение (\byr{}) \\
    \hline
    Основная заработная плата & $\text{З}_{\text{о}}$ & \num{\totalSalary} \\
    \hline
    Дополнительная заработная плата & $\text{З}_{\text{д}}$ & \num{\additionalSalary} \\
    \hline
    Отчисления в фонд социальной защиты & $\text{З}_{\text{сз}}$ &\num{\socialProtectionCost} \\
    \hline
    Затраты на материалы & $\text{М}$ &\num{\stuffCost} \\
    \hline
    Расходы на машинное время & $\text{Р}_{\text{м}}$ &\num{\machineTimeCost} \\
    \hline
    Расходы на командировки & $\text{Р}_{\text{к}}$ &\num{\businessTripCost} \\
    \hline
    Прочие затраты & $\text{П}_{\text{з}}$ &\num{\otherCost} \\
    \hline
    Накладные расходы & $\text{Р}_{\text{н}}$ &\num{\overheadCost} \\
    \hline
    Общая сумма расходов по смете & $\text{С}_{\text{р}}$ &\num{\overallCost} \\
    \hline
    Расходы на сопровождение и адаптацию & $\text{Р}_{\text{са}}$ &\num{\softwareSupportCost} \\
    \hline
    Полная себестоимость & $\text{С}_{\text{п}}$ &\num{\baseCost} \\
    \hline
    Прогнозируемая прибыль & $\text{П}_{\text{с}}$ &\num{\income} \\
    \hline
    НДС & $\text{НДС}$ &\num{\nds} \\
    \hline
    Прогнозируемая отпускная цена ПО & $\text{Ц}_{\text{п}}$ &\num{\sellingPrice} \\
    \hline
    Чистая прибыль & $\text{П}_{\text{ч}}$ &\num{\incomeWithTaxes} \\
    \hline
  \end{tabular}
\end{table}
\hfill

\subsection{Оценка экономической эффективности применения ПС у пользователя}

 Для расчета экономического эффекта применения нового ПС необходимы данные имеющегося внедренного на производстве аналога (базового варианта). Некоторые из показателей базового варианта не могут быть получены (составляют коммерческую тайну либо защищены авторскими правами разработчиков). Поэтому расчет экономического эффекта будем проводить, опираясь на известные данные показателей базового и нового варианта ПС. Показатели обоих вариантов и исходные данные для расчета экономического эффекта приведены в таблице~\ref{table:economic:compare_with_basic}

\begin{table}
  \caption{Исходные данные для расчета экономического эффекта}
  \label{table:economic:compare_with_basic}
  \centering
  \begin{tabular}{| >{\centering}m{0.25\textwidth}
                  | >{\centering}m{0.17\textwidth}
                  | >{\centering}m{0.13\textwidth}
                  | >{\centering}m{0.17\textwidth}
                  | >{\centering\arraybackslash}m{0.17\textwidth}|}
    \hline
           \multirow{2}{0.3\textwidth}{\centering Наименования}
         & \multirow{2}{0.17\textwidth}{\centering Обозначения}
         & \multirow{2}{0.13\textwidth}{\centering Eдиницы измерения}
         & \multicolumn{2}{c|}{\centering Значение показателя} \tabularnewline
    \cline{4-5} & &
         & { Базовый вариант }
         & { Новый вариант } \tabularnewline
     \hline
     Капитальные вложения & $\text{К}_{\text{пр}}$ & \byr{} & - & \num{\sellingPrice} \\
     \hline
     Затраты на освоение и сопровождение ПС & $\text{К}_{\text{ос}}$ & \byr{} & - & \num{\softwareSupportCost} \\
     \hline
     Затраты на пополнение оборотных фондов, связанных с эксплуатацией нового ПС & & \byr{} & - & \num{\fundsExploationSoftware} \\
     \hline
     Всего затрат & & \byr{} & & \num{\allCostsSoftware} \\
     \hline
     Время простоя сервиса, обсуловленного ПО, в день & $\text{П}_{\text{1}}$, $\text{П}_{\text{2}}$ & мин & \num{\baseStopTime} & \num{\newStopTime} \\
     \hline
     Стоимость одного часа простоя & $\text{С}_{\text{п}}$ & \byr{} & \num{\stopHourPrice} & \num{\stopHourPrice} \\
     \hline
     Среднемесячная зарплата одного программиста & $\text{З}_{\text{см}}$ & \byr{} & \num{\monthlyProgPayment} & \num{\monthlyProgPayment} \\
     \hline
     Коэффициент начислений на зарплату & $\text{К}_{\text{нз}}$ & - & \num{\paymentUpCoeff} & \num{\paymentUpCoeff} \\
     \hline
     Среднемесячное количество рабочих дней & $\text{Д}_{\text{р}}$ & день & \num{\averageWorkingDaysMontly} & \num{\averageWorkingDaysMontly} \\
     \hline
     Объем выполняемых работ & $\text{А}_{\text{1}}$, $\text{А}_{\text{2}}$ & задача & \num{\jobsVolume} & \num{\jobsVolume} \\
     \hline
     Средняя трудоемкость работ в расчете на задачу & $\text{Т}_{\text{с1}}$, $\text{Т}_{\text{с2}}$ & человеко-часов на задачу & \num{\baseAverageLabourIntensity} & \num{\newAverageLabourIntensity} \\
     \hline
   \end{tabular}
\end{table}

Общие капитальные вложения ($ \text{К}_{\text{о}} $) заказчика(потребителя), связанные с приобретением, внедрением и использованием ПС, включают в себя затраты на приобретение, освоение ПС, а также затраты на доукомплектацию техническими средствами в связи с внедрением нового ПС, а также затраты на пополнение оборотных средств в связи с использованием нового ПС.

Экономия затрат на заработную плату в расчете на 1 задачу ($ \text{С}_{\text{зе}} $) рассчитаем по формуле~\ref{eq:economic:spendingEconomyOnPayment}:
\begin{equation}
  \label{eq:economic:spendingEconomyOnPayment}
  \text{C}_{\text{зе}} =
    \frac { \text{З}_{\text{см}} \times (\text{Т}_{\text{с1}} - \text{Т}_{\text{с2}}) }
          { \text{Т}_{\text{ч}} \times \text{Д}_{\text{р}}}
\end{equation}
Экономия заработной платы при использовании нового ПО ($ \text{С}_{\text{з}} $)рассчитывается по формуле~\ref{eq:economic:spendingEconomyByNewSoftware}:
\begin{equation}
  \label{eq:economic:spendingEconomyByNewSoftware}
  \text{C}_{\text{з}} = { \text{З}_{\text{зе}} \times \text{А}_{\text{2}} }
\end{equation}
Экономия с учетом начисления на зарплату ($ \text{С}_{\text{н}} $) рассчитывается по формуле~\ref{eq:economic:spendingEconomyByUp}
\begin{equation}
  \label{eq:economic:spendingEconomyByUp}
  \text{C}_{\text{н}} = { \text{С}_{\text{з}} \times 1,5 }
\end{equation}
Экономия за счет сокращения простоев сервиса ($ \text{С}_{\text{с}} $) рассчитывается по формуле~\ref{eq:economic:spendingEconomyByStopReducing}:
\begin{equation}
  \label{eq:economic:spendingEconomyByStopReducing}
  \text{C}_{\text{с}} =
    \frac { (\text{П}_{\text{1}} - \text{П}_{\text{2}}) \times \text{Д}_{\text{рг}} \times \text{С}_{\text{п}} }
          { 60 }
\end{equation}
\begin{explanation}
  где & $ \text{Д}_{\text{рг}} $ & плановый фонд работы сервиса (дней).
\end{explanation}

Общая годовая экономия текущих затрат, связанных с использованием нового ПО ($ \text{C}_{\text{о}} $), рассчитывается по формуле~\ref{eq:economic:commonYearEconomy}:
\begin{equation}
  \label{eq:economic:commonYearEconomy}
  \text{C}_{\text{о}} = \text{C}_{\text{с}} + \text{C}_{\text{н}}
\end{equation}

Расчет капитальных затрат приведен в таблице~\ref{table:economic:commonPaymentCalculation}

\begin{table}
\caption{Расчет капитальных затрат}
\label{table:economic:commonPaymentCalculation}
  \centering
  \begin{tabular}{| >{\centering}m{0.5\textwidth}
                  | >{\centering}m{0.15\textwidth}
                  | >{\centering\arraybackslash}m{0.25\textwidth}|}
    \hline
    {\begin{center}
      Наименование
    \end{center} } & Условное обозначение & Значение (\byr{}) \\
    \hline
    Общие капитальные затраты & $\text{К}_{\text{о}}$ & \num{\allCostsSoftware} \\
    \hline
    Экономия затрат на заработную плату в расчете на 1 задачу & $\text{С}_{\text{зе}}$ & \num{\paymentEconomyOnOneTask} \\
    \hline
    Экономия затрат на заработную плату при использовании нового ПС & $\text{С}_{\text{з}}$ & \num{\paymentEconomyByUsingNewSoftware} \\
    \hline
    Экономия затрат c учетом начисления на заработную платну & $\text{С}_{\text{н}}$ & \num{\paymentEconomyCountingUp} \\
    \hline
    Экономия затрат за счет сокращения простоя сервиса & $\text{С}_{\text{с}}$ & \num{\paymentEconomyByReducingStopTime} \\
    \hline
    Общая годовая экономия текущих затрат, связанных с использованием нового ПС & $\text{С}_{\text{о}}$ & \num{\commonPaymentEconomyByUsingNewSoftware} \\
    \hline
  \end{tabular}
\end{table}

Внедрение нового ПС позволит сэкономить на текущих затратах \num{\commonPaymentEconomyByUsingNewSoftware}.

\subsection{Расчет экономического эффекта}

Для пользователя в качестве экономического эффекта выступает лишь чистая прибыль - дополнительная прибыль, остающаяся в его распоряжении, которая определяется по формуле~\ref{eq:economic:pure_income}

\begin{equation}
  \label{eq:economic:pure_income}
  \text{П}_{\text{ч}} = \text{С}_{\text{о}} -
    \frac { \text{С}_{\text{о}} \times \text{Н}_{\text{п}}}
          { 100 \% }
\end{equation}

Тогда чистая прибыль будет равна:

\begin{equation}
  \label{eq:economic:pure_income}
  \text{П}_{\text{ч}} = \text{45233,3} - (\text{45233,3} \times \text{18\%}) = \text{\pureIncome}
\end{equation}


В процессе использования нового ПС чистая прибыль в конечном итоге возмещает капистальные затраты. Однако полученные при этом суммы результатов (прибыли) и затрат (капитальных вложений) по годам приводят к единому времени - расчетному году (за расчетный год принят 2017 год) путем уменожения результатов и затрат за каждый код на коэффициент приведения, который рассчитывается по формуле~\ref{eq:economic:reduction_coeff}:

\begin{equation}
  \label{eq:economic:reduction_coeff}
  \alpha_{\text{1}} = (\text{1} - \textit{A}_{\textit{i}})^{\textit{ip}-\text{1}}
\end{equation}

Коэффициентам приведения по годам будут соответствовать значения:

\centerline{$ \alpha_{\text{1}} = \num{\reductionCoeffFirst} $ - расчётный год}
\centerline{$ \alpha_{\text{2}} = \num{\reductionCoeffSecond} $ - 2018 год}
\centerline{$ \alpha_{\text{3}} = \num{\reductionCoeffThird} $ - 2019 год}
\centerline{$ \alpha_{\text{4}} = \num{\reductionCoeffFourth} $ - 2020 год}

Расчёт экономического эффекта от использования нового программного средства представлен в таблице~\ref{table:economic:economicEffect}.

\begin{table}
\caption{Расчёт экономического эффекта от использования нового программного продукта}
\label{table:economic:economicEffect}
  \centering
  \begin{tabular}{| >{\centering}m{0.20\textwidth}
                  | >{\centering}m{0.11\textwidth}
                  | >{\centering}m{0.11\textwidth}
                  | >{\centering}m{0.11\textwidth}
                  | >{\centering}m{0.11\textwidth}
                  | >{\centering\arraybackslash}m{0.11\textwidth}|}
    \hline
    {\begin{center}
      Наименование
    \end{center} } & Единица измерения & 2017 & 2018 & 2019 & 2020 \\
    \hline
    Приоритет прибыли за счёт экономии затрат & \byr{} & - & \num{\pureIncome} & \num{\pureIncome} & \num{\pureIncome} \\
    \hline
    Прирост прибыли с учётом фактора времени & \byr{} & - & \num{\overcomeValueSecond} & \num{\overcomeValueThird} & \num{\overcomeValueFourth} \\
    \hline
    Затраты на освоение и сопровождение ПС & \byr{} & \num{\softwareSupportCost} & & & \\
    \hline
    Затраты на пополнение оборотных фондов, связанных с эксплуатацией нового ПС & \byr{} & \num{\fundsExploationSoftware} & & & \\
    \hline
    Всего затрат & \byr{} & \num{\allCostsSoftware}  & & & \\
    \hline
    То же с учётом фактора времени & \byr{} & \num{\allCostsSoftware}  & & & \\
    \hline
    Экономический эффект: & & & & & \\
    \hline
    Превышение результата над затратами: & \byr{} & -\num{\allCostsSoftware} & \num{\overcomeValueSecond} & \num{\overcomeValueThird} & \num{\overcomeValueFourth} \\
    \hline
    То же с нарастающим итогом: & \byr{} & -\num{\allCostsSoftware} & \num{\growingResultSecond} & \num{\growingResultThird} & \num{\growingResultFourth} \\
    \hline
    Коэффцициент приведения: & единиц & \num{\reductionCoeffFirst} & \num{\reductionCoeffSecond} & \num{\reductionCoeffThird} & \num{\reductionCoeffFourth} \\
    \hline
  \end{tabular}
\end{table}

\subsection{Выводы по технико-экономическому обоснованию}
В результате технико-экономического обоснования были рассчитаны следующие экономические показатели:

\begin{itemize}
  \item смета затрат составляет $ \num{\baseCost} \byr{}$ и отпускная цена составляет $ \num{\sellingPrice} \byr{}$;
  \item экономия затрат у пользователя составляет $ \num{\pureIncome} \byr{}$ и окупается на второй год;
\end{itemize}

Таким образом, разработка сервиса, предоставляющего возможность получения новинок музыкальной индустрии, является экономически целесообразной.
