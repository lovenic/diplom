\sectioncentered*{Реферат}
\thispagestyle{empty}

\begin{center}
  \begin{minipage}{0.90\textwidth}
    СЕРВИС, ПРЕДОСТАВЛЯЮЩИЙ ВОЗМОЖНОСТЬ ПОЛУЧЕНИЯ НОВИНОК МУЗЫКАЛЬНОЙ ИНДУСТРИИ: дипломный проект / Т.С. Лавник. -- Минск: БГУИР, 2017, -- п.з. -- ~\pageref*{LastPage} с., чертежей (плакатов) -- 6 л. формата A1.
  \end{minipage}
\end{center}

\emph{Ключевые слова}: МУЗЫКАЛЬНАЯ ПУБЛИКАЦИЯ, АРТИСТ, ТИП ПУБЛИКАЦИИ, ПОИСК АРТИСТОВ, ТРЕХУРОВНЕВАЯ АРХИТЕКТУРА.

Объектом исследования и разработки является сервис, представляющий возможность получения новинок музыкальной индустрии.

Целью дипломного проекта является создание инструмента, позволяющего решать задачи поиска новых музыкальных публикаций удобным способом.

Для достижения цели дипломного проекта было разработано веб-приложение на фреймворке Ruby on Rails.
Веб-приложение может быть использовано для получения новых публикаций интересующих пользователя артистов.
В приложении используются алгоритмы, позволяющие периодически проверять различные сервисы на наличие обновлений, касающихся интересующих релизов артистов. Также в приложении присутствует email-рассылка, отправляемая пользователям в случае наличия обновлений.

Во введении производится ознакомление с проблемой, решаемой в дипломном проекте.

В первой главе производится обзор предметной области проблемы решаемой в данном дипломном проекте.
Приводятся необходимые теоретические сведения, а также производится обзор существующих разработок.

Во второй главе производится краткий обзор технологий, использованных для реализации ПО в рамках дипломного проекта.

В третьей главе производится обзор реализованного ПО.
Описываются его составные части и особенности.
Приводятся результаты практических испытаний и производится сравнение с существующим ПО.

В четвертой главе производится технико"=экономическое обоснование разработки.

В заключении подводятся итоги и делаются выводы по дипломному проекту, а также описывается дальнейший план развития проекта.

Дипломный проект является завершенным, поставленная задача решена в полной мере, присутствует возможность дальнейшего развития приложения и увеличение его функциональности.

\clearpage
