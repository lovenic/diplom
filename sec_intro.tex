\sectioncentered*{Введение}
\addcontentsline{toc}{section}{Введение}
\label{sec:intro}

Музыка сопровождала человечество на протяжении всей истории. Вначале люди использовали только свой голос, потом появились музыкальные инструменты. До сих пор прогресс не стоит на месте. С появлением компьютерной техники и интернета появилось ещё больше возможностей для самовыражения: программное обеспечение для звукозаписи, веб-сайты для публикации своих произведений, сервисы для анализа предпочтений и тому подобное. Учитывая существование изобилия музыкальных ресурсов, которые предоставлют пользователям разнообразные возможности, становится трудно следить за новыми публикациями любимых исполнителей, и, хотя подобные сервисы уже присутствуют на мировом рынке, не все они являются удобными и логически законченными.

В связи с вышесказанным, целью данного диплома является создание сервиса, который будет предоставлять возможность удобного получения уведомлений о новинках музыкальной индустрии. Сервис должен быть оснащён понятным и дружелюбным интерфейсом, позволять пользователю удобно регистрироваться и наполнять данными свой профиль, указывая свои предпочтения в виде списка интересующих артистов.

Для того, чтобы сервис был полезен пользователю, он должен быть удобен, привлекателен и реализовывать некоторую потребность. В этом состоит основа данного дипломного проекта: собирать информацию из разных сервисов в одном месте, тем самым упрощая этот для конечного пользователя. В некотором роде - это агрегат, который собирает интересы и возвращает ответ в виде удовлетворения этих интересов путём указания ссылок на публикации любимых артистов. Удобство должно быть выражено в дизайне интерфейса, который должен быть интуитивно ясен пользователю и помогать ему производить некоторые операции внутри сервиса. Также проект должен быть хорошо масштабируемым, чтобы можно было легко добавлять разные источники публикаций в свой арсенал.

Дипломный проект выполнен самостоятельно, проверен в системе «Атиплагиат». Процент оригинальности соответствует норме, установленной кафедрой информатики. Цитирования обозначены ссылками на публикации, указанные в разделе «Список использованных источников».
