\sectioncentered*{Аннотация}
\thispagestyle{empty}

\begin{center}
  \begin{minipage}{0.82\textwidth}
    на дипломный проект <<Сервис, предоставляющий возможность получения новинок музыкальной индустрии>> студента УО <<Белорусский государственный университет информатики и радиоэлектроники>> Лавника~Т.\,С.
  \end{minipage}
\end{center}

\emph{Ключевые слова}: музыкальная публикация, артист, тип публикации, поиск артистов, трехуровневая архитектура.

\vspace{4\parsep}

Дипломный проект выполнен на 6 листах формата А1 с пояснительной запиской на~\pageref*{LastPage} страницах, без приложений справочного или информационного характера.
Пояснительная записка включает \total{section}~глав, \totfig{}~рисунков, \tottab{}~таблиц, \toteq{}~формул и \totref{}~литературный источник.

Целью дипломного проекта является создание инструмента, позволяющего решать задачи поиска новых музыкальных публикаций удобным способом.

Для достижения цели дипломного проекта было разработано веб-приложение на фреймворке Ruby on Rails.
Веб-приложение может быть использовано для получения новых публикаций интересующих пользователя артистов.
В приложении используются алгоритмы, позволяющие периодически проверять различные сервисы на наличие обновлений, касающихся интересующих релизов артистов. Также в приложении присутствует email-рассылка, отправляемая пользователям в случае наличия обновлений.

Во введении производится ознакомление с проблемой, решаемой в дипломном проекте.

В первой главе производится обзор предметной области проблемы решаемой в данном дипломном проекте.
Приводятся необходимые теоретические сведения, а также производится обзор существующих разработок.

Во второй главе производится краткий обзор технологий, использованных для реализации ПО в рамках дипломного проекта.

В третьей главе производится обзор реализованного ПО.
Описываются его составные части и особенности.
Приводятся результаты практических испытаний и производится сравнение с существующим ПО.

В четвертой главе производится технико"=экономическое обоснование разработки.

В заключении подводятся итоги и делаются выводы по дипломному проекту, а также описывается дальнейший план развития проекта.

\clearpage
